% !TeX spellcheck = ru_RU
\documentclass[a4paper, 14pt, unknownkeysallowed]{extreport}
\include{settings}

\begin{document}
\include{title}
\setcounter{page}{2}

\chapter{Теоретическая часть}
Случайный процесс называется марковским, если он обладает следующим свойством: для каждого момента $t_0$ вероятность любого состояния системы в будущем (при $t > t_0$) зависит только от ее состояния в настоящем (при $t = t_0$) и не зависит от того, когда и каким образом система пришла в это состояние.  

Вероятностью i-го состояния называется вероятность $p_i(t)$ того, что в момент t система будет находится в состоянии $S_i$. Также для любого времени t выполняется условие нормировки --- сумма вероятностей всех состоянии равна единице.

Стационарное распределение цепи Маркова --- распределение вероятности, которое не меняется с течением времени.

Предельная вероятность состояния показывает среднее время пребывания системы в этом состоянии. Для нахождения таких вероятностей используют уравнения Колмогорова.

Уравнение Колмогорова строится по следующему правилу: в левой части каждого уравнения стоит производная вероятности состояния, а правая часть содержит столько членов сколько стрелок связано с данным состоянием.

Если стрелка направлена из состояния, то соответствующий член имеет знак "$-$", в состояние --- "$+$". 

Каждый член равен произведению плотности вероятности перехода соответствующий данной стрелке, умноженной на вероятность того состояния, из которого исходит стрелка. 

\newpage
Получаем систему уравнений вида:

\begin{equation}
	\label{ass}
	\begin{cases}
		p_1'(t) =  -(\lambda_{11} + ... + \lambda_{1n})p_1 + \lambda_{21}*p_2+...+\lambda_{n1}*p_n\\
		p_2'(t) =  -(\lambda_{21} + ... + \lambda_{2n})p_2 + \lambda_{12}*p_1+...+\lambda_{n2}*p_n\\
		...\\
		p_n'(t) =  -(\lambda_{n1}+ ... + \lambda_nn)p_n + \lambda_{1n}*p_1+...+\lambda_{nn}*p_{n}.
	\end{cases}
\end{equation} 

В стационарном случае ($p_i'(t) = 0, i = \overline{1, n}$) левые системы уравнений принимают значение 0. Получаем СЛАУ, решение которой дает нам искомые предельные вероятности состояний системы. 

Интегрирование системы (\ref{ass}) дает искомые вероятности системы как функции времени. 
График зависимости вероятности пребывания системы в состоянии $p_i$ колеблется, а затем в определенный момент $t_i!$ принимает стабильное значение $p_i!$. Это время $t_i!$ до стабилизации вычисляется в реализованной программе для каждого $i = \overline{1, n}$.

\chapter{Результат}
На рисунке \ref{img:1} приведен пример работы программы.
\img{0.8}{1}{Пример работы программы}

В листингах \ref{lst:1} -- \ref{lst:2} приведена реализация вычислений коэффициентов уравнений Колмогорова, вычисления предельных вероятностей и времени до стабилизации.
\lst{9}{13}{func.py}{1}{Реализация программных вычислений}
\lst{14}{54}{func.py}{2}{Реализация программных вычислений (продолжение)}

\end{document}