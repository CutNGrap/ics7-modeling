% !TeX spellcheck = ru_RU
\documentclass[a4paper, 14pt, unknownkeysallowed]{extreport}
\include{settings}

\begin{document}
\include{title}
\setcounter{page}{2}

\chapter{Теоретическая часть}

\section{Закон появления сообщений}

Согласно заданию лабораторной работы для генерации сообщений используется равномерный закон распределения.
Случайная величина имеет равномерное распределение на отрезке $[a, b]$, если её функция плотности $p(x)$ имеет вид:

\begin{equation}
	\label{for:equal-1}
	p(x) = 
	\begin{cases}
		\frac{1}{b - a}, \text{если } x \in [a, b],\\
		0, \text{иначе.} \\
	\end{cases}
\end{equation}

Функция распределения $F(x)$ равномерной случайной величины имеет вид:

\begin{equation}
	\label{for:equal-2}
	F(x) = 
	\begin{cases}
		0, \text{если } x  < a, \\
		\frac{x - a}{b - a}, \text{если } a < x  b,\\
		1, \text{если } x > b. 
	\end{cases}
\end{equation}

Интервал времени между появлением $i$-ого и $(i - 1)$-ого сообщения по равномерному закону распределения вычисляется следующим образом:

\begin{equation}
	T_{i} = a + (b - a) \cdot R,
\end{equation}
где $R$ --- псевдослучайное число от 0 до 1.

\section{Закон обработки сообщений}

Согласно заданию лабораторной работы для генерации сообщений используется нормальный закон распределения.
Случайная величина имеет равномерное распределение на отрезке [a, b], если ее плотность распределения p(x) и функция распределения F(x) равны

\begin{equation}
	p(x) = \begin{cases}
		\frac{1}{b - a}, a\leq x\leq b;\\
		0, x < a или x > b.\\
	\end{cases},   
	F(x) =  \begin{cases}
		0, x < a;\\
		\frac{x-a}{b-a}, a\leq x\leq b;\\
		1, x > b.
	\end{cases}
\end{equation}

\chapter{Результат}
На рисунке \ref{img:ex} приведен пример работы программы.
\img{0.7}{ex}{Пример работы программы}


\end{document}