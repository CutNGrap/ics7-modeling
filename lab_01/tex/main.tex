% !TeX spellcheck = ru_RU
\documentclass[a4paper, 14pt, unknownkeysallowed]{extreport}
\include{settings}

\begin{document}
\include{title}
\setcounter{page}{2}

\chapter{Постановка задачи}
Цель работы: теоретическое изучение некоторых типовых законов распределения случайных величин.

Согласно 2 варианту по списку было выполнено изучение следующих законов распределения:
\begin{itemize}
	\item равномерное распределение на отрезке;
	\item нормальное распределение.
\end{itemize}


\chapter{Равномерное распределение}

Случайная величина имеет равномерное распределение на отрезке [a, b], если ее плотность распределения p(x) и функция распределения F(x) равны

\begin{equation}
	p(x) = \begin{cases}
		\frac{1}{b - a}, a\leq x\leq b;\\
		0, x < a или x > b.\\
	\end{cases},   
	F(x) =  \begin{cases}
		0, x < a;\\
		\frac{x-a}{b-a}, a\leq x\leq b;\\
		1, x > b.
	\end{cases}
\end{equation}

Графики функции плотности распределения p(x) и функции распределения F(x) в общем виде приведены на рисунках \ref{img:1}--\ref{img:2}.

\img{0.45}{1}{График функции плотности распределения p(x) случайной величины, распределенной равномерно на отрезке [a, b].}
\img{1.3}{2}{График функции  распределения F(x) случайной величины, распределенной равномерно на отрезке [a, b].}



Полученные в результате работы программы графики функции плотности распределения p(x) и функции распределения F(x) для разных значений параметров a и b приведены на рисунках \ref{img:3}--\ref{img:5}.

\img{0.6}{3}{Графики функции плотности распределения (слева) и функции распределения (справа) случайной величины, распределенной равномерно на отрезке [2, 5].}
\img{0.6}{4}{Графики функции плотности распределения (слева) и функции распределения (справа) случайной величины, распределенной равномерно на отрезке [1, 6].}
\img{0.6}{5}{Графики функции плотности распределения (слева) и функции распределения (справа) случайной величины, распределенной равномерно на отрезке [3, 5].}


\chapter{Нормальное распределение}

Случайная величина распределена по нормальному закону, или имеет нормальное распределение, если ее плотность

\begin{equation}
	\varphi_{m,\sigma}(x) = \frac{1}{\sigma \sqrt{2\pi}} e^{-\frac{(x-m)^2}{2\sigma^2}} (-\inf<m<+\inf, \sigma > 0).
\end{equation}

Функция нормального распределения имеет следующий вид:
\begin{equation}
	\Phi_{m, \sigma}(x) = \frac{1}{\sigma \sqrt{2\pi}} \int_{-\inf}^{x} e^{-\frac{(x-m)^2}{2\sigma^2}} dx.
\end{equation}

Полученные в результате работы программы графики функции плотности распределения $\varphi_{m,\sigma}(x)$ и функции распределения $\Phi_{m, \sigma}$ для разных значений параметров m и $\sigma$ представлены на рисунках \ref{img:6}--\ref{img:8}.

\img{0.6}{6}{Графики функции плотности распределения (слева) и функции распределения (справа) случайной величины, распределенной по закону N(0,~1).}
\img{0.6}{7}{Графики функции плотности распределения (слева) и функции распределения (справа) случайной величины, распределенной по закону N(2,~1).}
\img{0.6}{8}{Графики функции плотности распределения (слева) и функции распределения (справа) случайной величины, распределенной по закону N(0,~2).}
\end{document}