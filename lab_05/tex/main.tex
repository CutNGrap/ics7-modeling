% !TeX spellcheck = ru_RU
\documentclass[a4paper, 14pt, unknownkeysallowed]{extreport}
\include{settings}

\begin{document}
\include{title}
\setcounter{page}{2}

\chapter{Теоретическая часть}
В информационный центр приходят клиенты через интервал времени 10 +- 2 минуты. Если все три имеющихся оператора заняты, клиенту отказывают в обслуживании. Операторы имеют разную производительность и могут обеспечивать обслуживание среднего запроса пользователя за 20 +- 5; 40 +- 10; 40 +- 20. Клиенты стремятся занять свободного оператора с максимальной производительностью. Полученные запросы сдаются в накопитель. Откуда выбираются на обработку. На первый компьютер запросы от 1 и 2-ого операторов, на второй – запросы от 3-его. Время обработки запросов первым и 2-м компьютером равны соответственно 15 и 30 мин. Промоделировать процесс обработки 300 запросов. 
Необходимо для этого создать концептуальную модель в терминах СМО, определить эндогенные и экзогенные переменные и уравнения модели. За единицу системного времени выбрать 0,01 минуты.


В процессе взаимодействия клиентов с информационным центром возможно:
1) Режим нормального обслуживания, т.е. клиент выбирает одного из свободных операторов, отдавая предпочтение тому у которого меньше номер.
2) Режим отказа в обслуживании клиента, когда все операторы заняты


\chapter{Результат}
На рисунке \ref{img:ex} приведен пример работы программы.
\img{0.9}{ex}{Пример работы программы}


\end{document}