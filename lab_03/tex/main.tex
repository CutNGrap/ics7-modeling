% !TeX spellcheck = ru_RU
\documentclass[a4paper, 14pt, unknownkeysallowed]{extreport}
\include{settings}

\begin{document}
\include{title}
\setcounter{page}{2}

\chapter{Теоретическая часть}
Для выполнения работы был выбран критерий «хи-квадрат». Это один из самых известных статистических критериев,  также это основной метод, используемый в сочетании с другими критериями. 
С помощью этого критерия можно узнать, удовлетворяет ли генератор случайных чисел требованию равномерного распределения или нет. 
Для оценки по этому критерию необходимо вычислить статистику V по формуле:

\begin{equation}
	\label{ass}
	V = \frac{1}{n}\sum_{s=1}^{k}(\frac{Y^{2}_S}{p_s}) - n,
\end{equation} 

где n – количество независимых испытаний, k – количество категорий, $Y_s$ — число наблюдений, которые действительно относятся к категории S,  $p_s$ — вероятность того, что случайное наблюдение относится к категории s.  

Значение V является значением критерия «хи-квадрат» для экспериментальных данных. Приемлемое значение этого критерия можно определить по таблице на рисунке \ref{img:1.jpg}. Для этого используем строку с v = k-1, где k = 10, 90, 900 для задания лабораторной.  P в этой таблице — это вероятность того, что экспериментальное значение $V_e$ будет меньше теоретического $V_t$ или равно ему. Ее также можно рассматривать как доверительную вероятность.

Если вычисленное по таблице $P_V$ окажется меньше 0.01 или больше 0.99, можно сделать вывод, что эти числа недостаточно случайные. Если $P_V$ лежит между 0.01 и 0.05 или между 0.95 и 0.99, то эти числа «подозрительны». Если $P_V$ лежит между 0.05 и 0.1 или 0.9-0.95, то числа можно считать «почти подозрительными». Обычно необходимо произвести проверку три раза и более с разными данными. Если по крайней мере два из трех результатов оказываются подозрительными, то числа рассматриваются как недостаточно случайные.  

\img{0.5}{1.jpg}{Некоторые процентные точки $\chi^2$- распределения.
	(Источник: Кнут Д. Э. «Искусство программирования» )
}

\chapter{Результат}
На рисунке \ref{img:ex} приведен пример работы программы.
\img{0.6}{ex}{Пример работы программы}


\end{document}